\section{Background}

Maternal and perinatal health is a critical aspect of public health, and investigating the trends and risk factors associated with it is crucial for developing effective interventions. Max Roser et.al, in their article, posted online highlights the significant progress that has been made in reducing child mortality over the past few decades but also notes that there is still a long way to go, particularly in low-income countries and further improvements could be made to reduce the maternal and perinatal rates in developed countries.

During 1990 and 2015, the global maternal mortality ratio (MMR) at 44\% failed to meet the millennium development goal (MDG) 5 which was set to 75\% \cite{bongaarts_2016}.  The sustainable development goal (SDG) 3.1 and 3.2 aims to reduce the global maternal mortality ratio to less than 70 per 100,000 live births by 2030 and end preventable deaths of newborns and children under 5 years of age, with all countries aiming to reduce neonatal mortality to at least as low as 12 per 1,000 live births and under 5 mortality to at least as low as 25 per 1,000 live births. 

While progress has been made in reducing maternal and perinatal mortality rates globally, disparities still exist within and between countries. In Australia, stillbirth rates are higher than in many other developed countries, and maternal mortality rates have been relatively stable over the last decade \cite{owidchildmortality}. Australia has low maternal and perinatal mortality rates, but stillbirth remains a concern as neonatal deaths have mainly contributed to the decline in the perinatal rates. 

A systematic review conducted by Nik Hazlina NH et.al. on the prevalence of risk factors for maternal morbidity suggests that the risk factors of history of cesarean section, advanced maternal age, multiple pregnancies, co-existing medical conditions, and cesarean birth relative to planned vaginal deliveries, were associated with severe maternal morbidity (SMM) and maternal mortality \cite{nikkamil_2022,sauve_kramer_2007}. Factors like employment status, low household income, history of abortion, multiple births, and inadequate antenatal care also contributed to exacerbating the effect \cite{nikkamil_2022}.

In developed countries, the number of cesarean deliveries is increasing, and people believe that cesarean delivery is safer and better than vaginal delivery, leading to a change in the perceived risks and benefits. Cesarean deliveries, in the past, were mainly performed due to medical complications or serious illnesses, however, the rate of elective cesarean deliveries without any clear medical reasons has been increasing rapidly \cite{sauve_kramer_2007}.

Nik Hazlina NH et.al observed that studies conducted in both developed and developing countries show that younger adolescent girls, especially those aged 15 or younger, face higher mortality rates during childbirth than older adolescents. Adolescent pregnancies are more common among poor and less educated women, which disrupts their education and future career prospects \cite{nikkamil_2022}.

Further, it was found that twin pregnancies increase the risk of SMM and that twin pregnancies double the risk of SMM and increased the risk of maternal death four times compared to singleton pregnancies \cite{nikkamil_2022}.

HE Reinebrant et.al., mentions that one of the major challenges of curbing stillbirth is the lack of information on the causes of it. In low-income countries, stillbirths were often attributed to Infection, and Complications during labor and birth were often reported to be the cause of stillbirths in low-income countries while, placental complications were often attributed to stillbirths in middle- and high-income countries  \cite{reinebrant}. In high-income countries 32.1\% of stillbirth causes were unexplained, 14.4\% of stillbirths were caused by antepartum hemorrhage, 9.3\% were caused by placental conditions, 8.4\% caused by congenital anomalies, 14\% were caused by unspecified conditions and 22.7\% were caused by all other causes.

In developed countries, the prevalence of stillbirths is higher among mothers aged over 35 years, those who are overweight or obese, those with pre-existing hypertension, and those who consume tobacco-related products. Joy E Lawn et.al. states that detecting fetal growth restriction early remains a challenge even in high-income countries. In their study, they found that prolonged pregnancy beyond 42 weeks is linked to a higher risk of stillbirth, accounting for around 14\% of stillbirths worldwide and adolescent pregnancies, especially among those aged below 16 years, are also associated with an increased risk of stillbirth \cite{lawn_blencowe_2016}. 

Many studies have observed that smoking during pregnancy (SDP) is related to socioeconomic status, maternal age, partner smoking and cultural and ethnic background of the pregnant mother and have further demonstrated that SDP is associated with an increased risk of adverse pregnancy outcomes such as miscarriage, stillbirths, premature birth, low birth weight babies, neonatal morbidity, perinatal deaths, maternal mortality and even postnatal child health outcomes like sudden infant death syndrome, brain tumor, childhood cancer, hearing problem, asthma and other respiratory infection \cite{mohsin2010,jackson_2021,popova_2018}. The prevalence of SDP is around 10\% in high-income countries while 3\% in low- and middle-income countries \cite{jackson_2021}.

The systematic review conducted by Tuba Saygın Avşar et.al. found that smoking increased the risk for 20 conditions and the highest impact was observed for sudden infant death syndrome, asthma, LBW, stillbirth and obesity in infants. The study indicated that maternal smoking not only increased the risk of the child's death during the prenatal period, neonatal period and infancy but also affected some long-term outcomes which could be detrimental for offspring, for instance, some birth defects and intellectual disability would affect later stages of life \cite{jackson_2021,popova_2018}. 

Smoking during pregnancy is associated with an increased risk of adverse maternal and neonatal outcomes, including premature labor, premature rupture of membranes, preterm birth, intrauterine growth restriction (IUGR), and small for gestational age (SGA) infants. However, quitting smoking during the first trimester can reduce these risks \cite{clifton_2014,gibbons_2013}. 

Smoking rates have decreased in Australia, but rates among Aboriginal women are three times higher than those among non-Aboriginal women. Reducing maternal smoking during pregnancy is the most effective way to improve perinatal outcomes for Aboriginal women, but it is challenging due to socioeconomic disadvantage, low education levels, and stressful life circumstances. Regardless of smoking status, Aboriginal women had lower maternal age, higher rates of teenage pregnancies, and were more likely to reside in areas with low socio-economic indexes and be public hospital patients.

Indigenous women in Australia are at a higher risk of poor perinatal outcomes compared to non-Indigenous women, regardless of smoking status. This disparity persists even after adjusting for potential confounding factors. Factors such as health literacy, inequitable education, remote location, and limited access to healthcare contribute to the increased risk of adverse outcomes in Indigenous populations. Compared with non-Indigenous women, Indigenous women attend their first antenatal appointment later in pregnancy, make fewer antenatal visits, and receive inconsistent antenatal care services \cite{gibbons_2013}.

A comprehensive review conducted by Valentin Simoncic et.al. found that the literature suggests that various maternal, parental, and contextual characteristics can affect the health of fetuses and newborns. Social determinants, including factors like maternal age, education, marital status, pregnancy intention, and socioeconomic status, have been found to play a significant role in fetal and newborn health outcomes. These social factors are also linked to health inequalities, particularly for pregnant women.

It also suggests that the well-being of a child during the first thousand days is linked to the health status of the mother during pregnancy, as well as the conditions in which the mother lives and works. Maternal health during pregnancy, such as excessive gestational weight gain, gestational diabetes mellitus, and obesity, can have significant effects on the health of newborns, including preterm birth. Therefore, to ensure the healthy development of a child during this period, it is essential to have a healthy mother and a healthy pregnancy.

Numerous studies have found that social factors such as education, income, and employment status are associated with the lifestyle and behaviors of pregnant women, including diet, physical activity, smoking, and alcohol consumption. These behaviors can lead to physiological disorders, including obesity, gestational weight gain, and gestational diabetes, which can result in adverse pregnancy outcomes. Additionally, research has shown that social deprivation may impact prenatal care utilization due to a lack of information and financial difficulties. Socially disadvantaged women may face more challenges in finding relevant and understandable information about maternity services and may be less able to pay for healthcare for themselves and their infants. Unemployed women have been identified as a barrier to healthcare utilization, and a lack of health insurance is also associated with inadequate use of antenatal care. SDP decreases in women with higher education and employment and is likely to stop smoking during pregnancy \cite{talantikite_2022}.

In addition, research conducted in high-income countries indicates that women living in deprived communities experience worse pregnancy outcomes. These socioeconomic inequalities disproportionately affect women, impacting both their own health and the health of their children. A meta-analysis of studies conducted in the UK and ROI found that women in lower-level occupations or social classes had a 40\% higher risk of stillbirth, neonatal mortality, perinatal mortality, preterm birth, and low birth weight. Additionally, unemployment is associated with an increased risk of stillbirth, maternal mortality, and preterm birth \cite{thomson_2021}.

The project's goal is to gain a better understanding of the factors that impact the health outcomes of babies by examining how maternal characteristics are associated with baby outcomes. The study aims to identify areas where interventions are necessary to improve the health outcomes of both mothers and babies. Ultimately, the insights from this research can help inform policies and initiatives aimed at promoting better health outcomes for mothers and babies across Australia.